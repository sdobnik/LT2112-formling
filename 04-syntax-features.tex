% This material is copyright Simon Dobnik and made available under the
% Creative Commons Attribution 4.0 International License (CC-BY-SA)
% license http://creativecommons.org/licenses/by-sa/4.0/
% 
% Email: simon.dobnik@gu.se
% Web: http://dobnik.net/simon/teaching/shared/LT2112-formling/


\documentclass{beamer}

\usepackage{graphicx,hyperref,lingmacros,qtree,avm,fullname}



\logo{\includegraphics[height=0.5cm]{pics/GU-logo.pdf}} 

\newcommand{\bblue}[1]{{\usebeamercolor[fg]{frametitle}{#1}}}



\definecolor{links}{HTML}{2A1B81}
\hypersetup{colorlinks,linkcolor=,urlcolor=links}


\setbeamertemplate{footline}[frame number]



\resetcounteronoverlays{enums}



\AtBeginSection[]
{
\begin{frame}[plain]

{\huge\bblue{\insertsectionhead}}

\end{frame}
}




\avmfont{\sc}
\avmsortfont{\scriptsize\it}

\renewcommand{\rmdefault}{\sfdefault}
\renewcommand{\scdefault}{\sfdefault}



\newcommand{\lb}[1]{[$_{\textsf{#1}}$} 
\newcommand{\ibar}[1]{$\overline{#1}$} 






\title[Syntax]{L4: Syntax II - Encoding grammar with features}

\author[Dobnik]{Simon Dobnik \\
Department of Philosophy, Linguistics, and Theory of Science
} 

\date{September 17, 2015}





\begin{document}



\frame{\titlepage}





\frame{

\frametitle{Previously\ldots}

We developed a (simplified) CFG fragment of English.

\bigskip

But structural relations are not enough.

\begin{enumerate}

\item \pause \bblue{Selectional restrictions of constituents:}\\
	\begin{itemize}
		\item \emph{Agreement:} George/They likes/like butterflies.
		\item \emph{Sub-categorisation:} Alex liked */the park.
	\end{itemize}

\item \pause \bblue{``Random'' laws of human language:} \\Sentences must have subjects: It rains.

\item \pause \bblue{Sentence meaning:} \\\#The tree climbed up George.

\end{enumerate}

}





\frame{

\frametitle{Outline}

\tableofcontents

}





\section{Agreement}

\subsection{Examples of agreeing constituents}

\frame{

\frametitle{Subject-verb agreement}


\ldots in \bblue{Person} and \bblue{Number}


\eenumsentence{
   \item \lb{NP}George] \bblue{likes} to travel by plane.
   \item \lb{NP}Cats] \bblue{like} to travel by plane.
   }

\pause

\eenumsentence{
   \item \bblue{Does} \lb{NP}this cat] \bblue{like} to travel by plane?
   \item \bblue{Do} \lb{NP} these cats] \bblue{like} to travel by plane?
   }

\pause


\eenumsentence{
   \item *\lb{NP}Cats] \bblue{likes} to travel by plane.
   \item *\bblue{Do} \lb{NP}George] \bblue{likes} to travel by plane?
   }
}





\frame{

\frametitle{Main verb and auxiliary verb}

\ldots agree for \bblue{Tense} in English.

\eenumsentence{
   \item Many flights stop\bblue{ed} in Chicago.
   \item Many flights \bblue{did} stop in Chicago.
   \item \bblue{Did} any flights stop in Chicago?
   \item \bblue{Did} George stop in Chicago?
   }

}




\frame{

\frametitle{Case}

In Kambera the verb agrees with both subject and object.

\bigskip

\eenumsentence{
   \item \shortex{5}{[I Ama]$_s$ & na$_s$-kei-ya$_o$ & [na & r\'{i} & muru]$_o$}
     {the father & \textsc{3sg,nom}-buy-\textsc{3sg,acc} & the & vegetable & green}
     {``Father buys the green vegetables.''}
   \item \shortex{1}{Na$_s$-kei-ya$_o$}
     {\textsc{3sg,nom}-buy-\textsc{3sg,acc}}
     {``He/she buys it.''}
   }

\hfill \cite{Tallerman:2011aa}, p.159

}





\frame{

\frametitle{Case}

English pronouns behave similarly\ldots


\bigskip

\eenumsentence{
	\item She$_{nom}$ met her$_{acc}$.
	\item *She$_{nom}$ met he$_{nom}$.
	\item  *Him$_{acc}$ met she$_{nom}$.
	}


}





\frame{

\frametitle{Gender}

\bblue{Object-verb gender in Arabic}

\eenumsentence{
	\item uhibbuka\\
	I love you - said to a male.
	\item uhibbuki \\
	I love you - said to a female.
	}

\pause

\bblue{Subject-verb agreement in Portuguese}

\eenumsentence{
	\item Muito obrigad\bblue{o}. \\
	I am very grateful - said by a male.
	\item Muito obrigad\bblue{a}. \\
	I'm very grateful - said by a female.
	}
}





\frame{

\frametitle{Gender}

\bblue{Determiner-adjective-noun agreement in German}


\eenumsentence{
	\item de\bblue{r} Krach \\ the noise
	\item de\bblue{r} laut\bblue{e} Krach \\ the loud noise
	\item ein laut\bblue{er} Krach \\ a loud noise
        \item laut\bblue{er} Krach \\ loud noise
	}

\pause

\bblue{Interaction with Case}

\eenumsentence{
\item di\bblue{e} klug\bblue{e} stark\bblue{e} Frau \\ the clever strong woman (nom)
	\item de\bblue{r} klug\bblue{en} stark\bblue{en} Frau \\ the clever strong woman (gen)
	}

}





\frame{

\frametitle{Quantifiers}

\eenumsentence{
  \item Many letters have arrived.
  \item *Much letters have arrived.
  \item Not much post has arrived.
  \item *Not many post has arrived.
  }

}





\frame{

\frametitle{Why natural languages have agreement?}

\pause

Marking grammatical relations between constituents. 

\bigskip

From Northern Sotho, a Bantu language \cite{Tallerman:2011aa}, p.160:


\eenumsentence{\item \shortex{3}
     {Mp\v{s}a & \bblue{e}-lomile & ngwana.}
     {dog & \textsc{subj}-bit & child}
     {``The dog bit a/the child.''}    \item \pause \shortex{3}
     {Ngwana & mp\v{s}a & \bblue{e-mo}-lomile.}
     {child & dog & \textsc{subj-obj}-bit}
     {``As for the child, the dog bit him/her.''}
   \item \pause \shortex{3}
     {Mp\v{s}a & ngwana & \bblue{e-mo}-lomile.}
     {dog & child & \textsc{subj-obj}-bit}
     {``As for the dog, it bit the child.''}
   }




}





\frame{

\frametitle{How to represent agreement - a solution?}

Encode the agreement information in CFG rules.

\enumsentence{TP $\rightarrow$ NP T VP}

\pause

\eenumsentence{
	\item TP $\rightarrow$ NP3sg T3sg VPNon3sg
	\item TP $\rightarrow$ NPNon3sg TNon3sg VPNon3sg
	\item T3sg $\rightarrow$ does $|$ has $|$ \ldots
	\item TNon3sg $\rightarrow$ do $|$ have $|$ \ldots
	\item V3sg $\rightarrow$ likes $|$ \ldots
	\item VNon3sg $\rightarrow$ like $|$ $\ldots$
	\item \pause NP3sg $\rightarrow$ (D) (AP+) N3sg (PP+)
	\item NPNon3sg $\rightarrow$ (D) (AP+) NNon3sg (PP+)
	\item N3sg $\rightarrow$ George $|$ tree $|$ he $|$ she $|$ it $|$ \ldots 	\item NNon3sg $\rightarrow$ I $|$ you $|$ we $|$ they $|$ cats $|$ trees \ldots 	}


}





\section{Sub-categorisation}

\subsection{Verbs as predicates}

\frame{

\frametitle{The categories of verbs}

Currently, our VP rule is very general\ldots

\enumsentence{VP $\rightarrow$ (AdvP+) V (NP) (\{NP/CP\}) (AdvP+) (PP+) (AdvP+)}

\pause

\bigskip

But not every verb is compatible with every phrase that the rule generates. 

\bigskip

\eenumsentence{
	\item *George sleeps the birds.
	\item *George bought that Lydia sang an aria.
	}

}





\frame{

\frametitle{Different kinds of lexical incompatibility}

\eenumsentence{
	\item \#George typed a neutrino.
	\item \#The apple is eating the colourless green cat furiously.
	\item \pause *George said.
	\item *George kissed Lydia Simon.
	}


}





\frame{

\frametitle{Verbs as predicates}

\begin{enumerate}

\item \bblue{No argument} \\
\lb{NP}It] rains. \lb{NP}It] snows. \\
Sentences have to have subjects: dummy ``it''.

\item \pause \bblue{An agent argument: intransitive verbs} \\
\lb{NP}George] ran.

\item \pause \bblue{A theme argument: un-accusative verbs} \\
\lb{NP}George] fell.\\
\emph{Remember:} sentences have to have subjects!

\item \pause \bblue{An agent and a theme argument: transitive verbs}\\
\lb{NP}George] kissed \lb{NP} Lydia] again.
  
\end{enumerate}

}





\frame{

\begin{itemize}

\item[5.] \bblue{An agent, a theme, and a goal argument}\\
\lb{NP}George] gave \lb{NP}Lydia] \lb{NP}a present].\\
\lb{NP}George] gave \lb{NP}a present] \lb{PP}to Lydia]. 

\item[6.] \pause \bblue{An agent and a proposition argument: factives}\\
\lb{NP}George] said \lb{CP}that he enjoyed the opera].

\item[7.] \pause \bblue{An agent and an event argument}\\
\lb{NP}George] intended \lb{TP}to climb a tree].

\item[8.] \pause \bblue{An agent, a theme, and a proposition argument}\\
\lb{NP}George] asked \lb{NP}Lydia] \lb{CP}whether she wants to\ldots].

\item[9.] \pause \bblue{An agent, a theme and an event argument}\\
\lb{NP}George] wanted \lb{NP}Lydia] \lb{TP}to sign an aria].

\item[10] \pause \ldots

\end{itemize}

}





\frame{

\frametitle{Arguments are of types: Theta/$\theta$-roles}

\bblue{\href{http://en.wikipedia.org/wiki/Theta_role}{Thematic roles}}

\enumsentence{Agent, Theme, Goal\ldots}



\pause \bigskip

\bblue{Sub-categorisation frame/valency:} a collection of arguments/theta roles associated with a verb:

\enumsentence{give(Agent,Theme,Goal)}

\pause \bigskip

\bblue{Theta-hierarchy}

\enumsentence{Agent $<$ Experiencer $<$ Goal/Source/Location $<$ Theme}


\eenumsentence{
   \item Captain Alex sank the ship.
   \item The ship sank.    }

}





\frame{

\frametitle{Arguments are of types: Theta/$\theta$-roles}

A single verb can have different subcat-frames:


\eenumsentence{
	\item George found a snake.
	\item George found me a good song on iTunes.
	}


}





\frame{

\frametitle{Sub-categorisation resources for NLP}

\bblue{FrameNet:} an annotated lexical resource of linguistic predicates (frames) and their arguments (frame elements).

\bigskip

\href{http://framenet.icsi.berkeley.edu/fndrupal/}{http://framenet.icsi.berkeley.edu}
\href{http://spraakbanken.gu.se/eng/swefn}{http://spraakbanken.gu.se/eng/swefn}

}





\frame{

\frametitle{Communication}

Core frame elements:

\begin{enumerate}
\item \bblue{Communicator:} \underline{He} finds it hard to communicate\ldots
\item \bblue{Medium:} Opinions are usually communicated \underline{over the telephone}\ldots
\item \bblue{Message:} How do you communicate to them \underline{that you really like them}?
\item \bblue{Topic:} Had someone communicated to the capital \underline{about the disregard of the religious law}?
\item \bblue{Addressee:} The company must be able to communicate \underline{to potential customers}\ldots
\item \bblue{Amount of information:} He never really \underline{fully} communicated his intentions.
\item \bblue{Duration, frequency, manner, means, place, purpose, time\ldots}
\end{enumerate}


}





\subsection{Nouns as predicates}

\frame{

\frametitle{Nouns as predicates}

\eenumsentence{
	\item \lb{NP} a student \lb{PP}of linguistics] \\ \lb{PP}with long hair] ].
	\item \pause *\lb{NP} a student \lb{PP}of linguistics] \lb{PP}of physics] ]
	\item \lb{NP}a student \lb{PP}with short hair] \lb{PP}with blue top] \lb{PP}on a bike] ]
	\item \pause *\lb{NP}a student \lb{PP}with short hair] \lb{PP}of linguistics] ] ]
	\item \pause \lb{NP}the one \lb{PP}with short hair] ]
	\item *\lb{NP}the one \lb{PP}of linguistics] ] 
	}

\pause

\bigskip

Sub-categorisation arguments (\bblue{Complements}) are obligatory, other modifiers are optional (\bblue{Adjuncts}).

}





\frame{

\frametitle{Complements and adjuncts structurally}

\footnotesize

\Tree [.DP [.D the ]  [.NP [.AP [.A smart ] ] [.NP [.NP [.N student ] [.PP [.P of ] [.NP [.N linguistics ] ] ] ] [.PP [.P with ] [.NP [.AP [.A long ] ] [.NP [.N hair ] ] ] ] ] ] ]

}





\frame{

\frametitle{Complements and adjuncts structurally}

Sentences (TPs) make a similar tree. 

\bigskip

What takes the D's place?

\bigskip

What takes the AP's and PP's place?

}





\frame{

\frametitle{Can structural relations be generalised?}

\Tree [.XP AP\\Specifier\\\bblue{Agreement} [.XP [.XP X\\Head YP\\Complement\\\bblue{Argument} ] ZP\\Adjunct\\\bblue{Optional modifier} ] ]

}





\frame{

\frametitle{The X-bar theory}

\href{http://en.wikipedia.org/wiki/X-bar_theory}{The X-bar theory} \cite{Chomsky:1970aa,Jackendoff:1977aa}:

\eenumsentence{
	\item XP $\rightarrow$ Spec; \ibar{X}
	\item (\ibar{X} $\rightarrow$ \ibar{X}; ZP)
	\item \ibar{X} $\rightarrow$ X; YP
	}

}





\frame{

\frametitle{Representing sub-categorisation in CFG}


\begin{itemize}

\item A new rule for each predicate category? \\
  \begin{enumerate}
     \item VP$_{intran}$ $\rightarrow$ NP V$_{intran}$
     \item VP$_{tran}$ $\rightarrow$ NP V$_{tran}$ NP 
     \item \ldots
     \end{enumerate}

\item \pause Doable on its own but remember there is also agreement!


\item \pause   \bblue{Parametrise} each node in the tree with \bblue{feature structures}.

\end{itemize}

}





\section{Representing constraints: features and unification}

\subsection{Feature structures and unification}

\frame{

\frametitle{Feature structures and unification}

\begin{itemize}

\item Represent each node of a CFG tree as a FS.

\item \pause Compose the FSs in the same way as you compose the tree.

\item \pause Add constraints on FSs to rules (constraint-based formalism).

\item \pause Agreement, sub-categorisation, long distance dependencies, semantics, anything really\ldots

\end{itemize}

\pause \bigskip

\enumsentence{S $\rightarrow$ NP VP \\
The number of the NP is equal to the number of the VP.}

}





\frame{

\frametitle{Feature structures as attribute-value matrices (AVMs)}

\begin{avm}
\[ Cat & np \\
Number & sg \\
Person & 3 
\]
\end{avm}
\begin{avm}
\[ Cat & np \\
Agreement & \[ Number & sg \\
			Person & 3 \] 
\]
\end{avm}


\bigskip


\begin{itemize}

\item Lists of feature=value pairs

\item \bblue{Features/attributes:} atoms

\item \bblue{Values:} atoms or feature structures 
\item \bblue{Feature path:} a list of features through a FS leading to a value: $\langle$\textsc{Agreement} \textsc{Person}$\rangle$

\item Feature paths as \bblue{directed acyclic graphs} (DAGs).

\end{itemize}

}





\frame{

\frametitle{Shared or reentrant feature structures}

\begin{avm}
\[ Cat & tp \\
Head &	\[ Agreement	\@{1} \[	Number & sg \\
							Person & 3 \] \\
		  Subject			  \[ Agreement \@{1} \] \\
		  \] \\
\]
\end{avm}

}





\frame{

\frametitle{Unifying FSs}

\begin{itemize}

\item \bblue{Unification ($\sqcup$):} combine two FSs so that the resulting FS contains all the information from the original two, nothing more.

\item \bblue{Partial operation:} may be undefined (unlike union of sets).


\end{itemize}

\bigskip

{\footnotesize
F1: \begin{avm}
\[ Cat & np \\
Agreement & \[ Number sing \] \\
\]
\end{avm}
F2: \begin{avm}
\[ Cat & np \\
Agreement & \[ Person 3 \] \\
\]
\end{avm}

\bigskip

F1 $\sqcup$ F2 = \pause F3: \begin{avm}
\[ Cat & np \\
Agreement & \[ Number & sing \\
			Person & 3 \] \\
\]
\end{avm}
}

}





\frame{

\frametitle{Examples of unification}

{\footnotesize
F4: \begin{avm}
\[ Cat & np \]
\end{avm}
\hspace{2cm} F5: \begin{avm}
\[ Cat & vp \]
\end{avm}

\bigskip

F4 $\sqcup$ F5 = \pause F6: Undefined!

\bigskip

\pause

F7: \begin{avm}
\[ Cat & np \]
\end{avm}
\hspace{2cm}
F8: \begin{avm}
\[ Cat & \[ \]\\
 \]
\end{avm}

\bigskip

F7 $\sqcup$ F8 = \pause F9: \begin{avm}
\[ Cat & np \]
\end{avm}

\bigskip

\pause

F10: \begin{avm}
\[ \]
\end{avm}

\bigskip

F7 $\sqcup$ F10 = \pause F7 = \begin{avm}
\[ Cat & np \]
\end{avm}


}

}





\frame{

\frametitle{Examples of unification}

{\footnotesize

F11: \begin{avm}
\[ Agreement & \[ Number & sing \] \\
Subject & \[ Agreement & \[ Number & sing \\
					Person & 3 
					\] \\
		\] \\
\]
\end{avm}

\bigskip

F12: \begin{avm}
\[ Agreement & \@{1} \[ Number & sing 
				\] \\
  Subject & \[ Agreement & \@{1}
  				\] \\
\]
\end{avm}

\bigskip

F11 $\sqcup$ F12 = \pause F13: \begin{avm}
\[ Agreement & \@{1} \[ Number & sing \\
				    Person & 3
				\] \\
  Subject & \[ Agreement & \@{1}
  				\] \\
\]
\end{avm}
}


}





\subsection{Agreement in unification grammar}

\begin{frame}[fragile]

\frametitle{Agreement in unification grammar}


\eenumsentence{
   \item George \bblue{likes} planes.
   \item Cats \bblue{like} planes.
   }

\pause \bigskip

S $\rightarrow$ NP VP \\
$\langle$NP \textsc{Agreement}$\rangle$ = $\langle$VP \textsc{Agreement}$\rangle$ \\
\cite{Jurafsky:2009uq} and \cite{Shieber1986}


\pause \bigskip



\verb+S -> NP[AGR=?n] VP[AGR=?n]+

\bigskip

?n in FS is a \bblue{variable}!

\end{frame}





\begin{frame}[fragile]

\frametitle{Terminal nodes}


\bblue{Propagating features from terminals to heads}


\bigskip

\verb+V[AGR=[NUM=pl]] -> 'like'+

\verb+V[AGR=[NUM=sg,PERS=3]] -> 'likes'+

\verb+V[AGR=[]] -> 'liked'+

\pause \bigskip

\verb+N[AGR=[NUM=sg]] -> 'George'+

\verb+N[AGR=[NUM=pl]] -> 'planes' | 'cats'+

\verb+N[AGR=[]] -> 'sheep' | 'fish'+


\end{frame}





\begin{frame}[fragile]

\frametitle{Propagating features from heads to phrases}

\bblue{Head features:} features of a \bblue{head} that get propagated to the phrase 
\pause \bigskip

\verb+VP[AGR=?agr] -> V[AGR=?agr] NP+



\pause \bigskip

All head features can be grouped under the head FS and then propagated with a rule like 

\bigskip

\verb+X[HEAD=[CAT=v,AGR=[NUM=sg,PERS=3]],...]+

\verb+XP[HEAD=?head] -> ... X[HEAD=?head] ...+


\end{frame}





\subsection{Sub-categorisation in unification grammar}

\begin{frame}[fragile]

\frametitle{Sub-categorisation in unification grammar}


\eenumsentence{
   \item \lb{NP}George] intended \lb{TP}to climb a tree].
   \item \lb{NP}George] saw \lb{NP}Lydia].
   }

\pause \bigskip

Add a \textsc{subcat} feature to predicates in the lexicon.

\bigskip

\verb+V[HEAD=[...],SUBCAT=to_clause] -> intended+

\verb+V[HEAD=[...],SUBCAT=trans] -> saw+

\pause \bigskip

\verb+VP[HEAD=?head,SUBCAT=trans] -> +

\verb+   V[HEAD=?head,SUBCAT=trans] NP+

\verb+VP[HEAD=?head,SUBCAT=to_clause] -> +

\verb+   V[HEAD=?head,SUBCAT=to_clause] TP[-TENSE]+

\end{frame}





\frame{

\frametitle{More examples in NLTK}

nltk\_data/grammars/book\_grammars/

\bigskip

feat0.fcfg, feat1.fcfg and german.fcfg

}





\frame{ 
\frametitle{Further reading}

\begin{itemize}

\item \cite{Allen:1995aa}: Chapters 4.1 (Feature systems and augmented grammars), 4.2 (Some basic feature systems for English) and 4.3 (A simple grammar using features)


\item \cite{Bird:2009ab}: \href{http://www.nltk.org/book/ch09.html}{Chapter 9} Building Feature Based Grammars.

\item \cite{Jurafsky:2009uq}: Chapter 15.1 (Feature Structures), 15.2 (Unification feature structures), 15.3 (Feature structures in the grammar) and 15.4 (Implementation of Unification)

\item \cite{Shieber1986}: \href{http://dash.harvard.edu/handle/1/11576719}{http://dash.harvard.edu/handle/1/11576719}

\end{itemize}

}





\begin{frame}[allowframebreaks]{References}

\small


\bibliographystyle{fullname}
\bibliography{bibliography}

\end{frame}





\end{document}
